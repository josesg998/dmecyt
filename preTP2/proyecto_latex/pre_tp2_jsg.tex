\documentclass{article}

\usepackage{template}



\usepackage[utf8]{inputenc} % allow utf-8 input
\usepackage[T1]{fontenc}    % use 8-bit T1 fonts
\usepackage{hyperref}       % hyperlinks
\usepackage{url}            % simple URL typesetting
\usepackage{booktabs}       % professional-quality tables
\usepackage{amsfonts}       % blackboard math symbols
\usepackage{nicefrac}       % compact symbols for 1/2, etc.
\usepackage{microtype}      % microtypography
\usepackage{xcolor}         % colors
\usepackage{graphicx}
\usepackage{float}

\graphicspath{ {imagenes/} }

\title{Pre TP2: Data Mining en Ciencia y Tecnología}

\author{%
  José Saint Germain\\
  \texttt{josesg998@gmail.com} \\
}

\begin{document}

\maketitle

\section{Introducción}

El objetivo de este trabajo es familiarizarse con la generación de grafos
que representen un conjunto de datos. A su vez, se buscará visualizar,
manipular y comparar distintos tipos de grafos. Por último, se calculará
parámetros básicos de un grafo y se los comparará con modelos de redes
aleatorios (random), small world y libre de escala (scale-free).

\section{Métodos}
Nuestra fuente de trabajo es una base de datos de 18 sujetos en
4 estadíos del sueño (despierto + 3 estadíos del sueño). A su vez, para cada 
sujeto y estadío de sueño se utilizó una matriz de correlaciones de tamaño 
116x116 con correlaciones entre entre las señales BOLD de 116 reciones
cerebrales. En este artículo se trabajará con la matriz del sujeto número 12
en estado despierto (w).


Para procesar y visualizar los datos se utilizó el lenguaje de programación
Python, del cual usamos las siguientes librerías: Numpy, para procesamiento 
de matrices; Pandas, para generación y procesamiento de datos tabulares; 
Matplotlib y seaborn, para visualización de datos y Networkx, para generación 
y manipulación de grafos.

\section{Resultados y discusión}

De manera exploratoria, se visualizó la matriz de correlaciones del sujeto 12
en estado despierto de manera pesada y binarizada.

\begin{figure}[H]
  \centering  
  \includegraphics[width=1\textwidth]{1_Sujeto2W.png}
  \caption{Matrices de correlación del cerebro del sujeto 2 despierto}
\end{figure}

La imagen de la izquierda muestra la matriz pesada, mientras que la de la 
derecha muestra la matriz binarizada con un umbral de 0.08. De esa manera,
las correlaciones menores a ese umbral se muestra como negros y los mayores
como blancos. De esta manera, se puede observar con claridad las correlaciones
con valores bajos de toda la matriz.

Para continuar con el análisis, transformarmos esta matriz en un grafo. El
grafo resultante es una totalmente conectado, es decir que 

\begin{table}[h]
  \centering
  \begin{tabular}{lr}
      \toprule
      Métrica & Valor \\
      \midrule
      Distancia media & 1.209295 \\
      Grado promedio (K) & 11.500000 \\
      Nodo con grado máximo (kmax) & 36.000000 \\
      Coeficiente de clustering promedio (C) & 0.556270 \\
      Eficiencia & 0.296510 \\
      \bottomrule
      \end{tabular}
  \caption{Métricas del grafo G}
  \label{tab:parametros}
\end{table}

Subsecuentemente, observaremos el grafo simulando la forma de un cerebro,
ubicando cada nodo en su ubicación correspondiente. Adicionalmente, agregamos
un histograma de los grados de cada nodo.

\begin{figure}[H]
  \centering  
  \includegraphics[width=1\textwidth]{2_Sujeto2W.png}
  \caption{Visualización del grafo G}
\end{figure}

\section{Conclusión}

\section{Bibliografía}
\end{document}