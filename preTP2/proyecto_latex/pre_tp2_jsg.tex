\documentclass{article}
\usepackage[spanish]{babel}
\usepackage[utf8]{inputenc}
\usepackage{template}



\usepackage[utf8]{inputenc} % allow utf-8 input
\usepackage[T1]{fontenc}    % use 8-bit T1 fonts
\usepackage{hyperref}       % hyperlinks
\usepackage{url}            % simple URL typesetting
\usepackage{booktabs}       % professional-quality tables
\usepackage{amsfonts}       % blackboard math symbols
\usepackage{nicefrac}       % compact symbols for 1/2, etc.
\usepackage{microtype}      % microtypography
\usepackage{xcolor}         % colors
\usepackage{graphicx}
\usepackage{float}


\graphicspath{ {imagenes/} }

\title{Pre TP2: Data Mining en Ciencia y Tecnología}

\author{%
  José Saint Germain\\
  \texttt{josesg998@gmail.com} \\
}

\begin{document}

\maketitle

\section{Introducción}

El análisis de la topología de grafos (es decir, redes) es un área de investigación que
atañe a diferentes campos de estudio. Para ejemplificar el uso de grafos nos enfocaremos en
el los datos obtenidos en el trabajo de Tagliazucchi y colaboradores (2013) que busca rela-
cionar cambios en la modularidad de las redes construidas a partir de la señal de resonancia
magnética funcional (fMRI) con los distintos estadíos del sueño.
\section{Objetivos}
Familiarizarse con la generación de grafos que representen un conjunto de datos. Visua-
lizar, manipular y comparar distintos grafos. Calcular parámetros básicos de un grafo, y
compararlos con modelos de redes random, small world y scale-free.

\section{Estructura de los Datos}
En la carpeta DataSujetos se encuentran los archivos separados por cada sujeto y estadio. Para cada sujeto y estadío de sueño encontraremos una matriz de correlaciones de tamaño
116x116 con las correlaciones entre las señales BOLD de 116 regiones cerebrales.

Además se incluyen los nombres y coordenadas de las 116 regiones en un archivo apar-
te: aalextendedwithCoords.csv. Estas regiones están definidas a partir del atlas Automatic
Anatomical Labeling (AAL). Ejemplos de los procedimiento para comenzar el análisis pueden encontrarse en \href{https://colab.research.google.com/drive/1xU8p_YSeSxPAODgiyJwJAVuuDC-jrTwP#scrollTo=VG4joS9_OZCA#offline=true&sandboxMode=true}{este colab}.

\section{Preprocesamiento de los datos}

\textit{Cargar el dataset con los datos para cada sujeto y los nombres y coordenadas 
de las regiones cerebrales a las que se les registró la actividad. Reportar cuántos sujetos y cuántos estados de sueño se observan en el conjunto de
datos.}

\begin{table}[h]
  \centering
  \caption{Conteo de estados de sueño en el conjunto de datos.}
  \label{tab:estados_sueno}
  \begin{tabular}{lr}
    \toprule
    estado & cantidad de sujetos \\
    \midrule
    N1 & 18 \\
    N2 & 18 \\
    N3 & 18 \\
    W & 18 \\
    \bottomrule
  \end{tabular}
\end{table}

Como se observa en el cuadro 1, el dataset cuenta con 18 sujetos, teniendo cada uno lo
cuatro diferentes estados de sueño.

\section{Manipulación de datos}

\textit{5.1 Graficar la matriz de correlaciones entre regiones (es decir, la "matriz de adyacencia
pesada") para el sujeto 2 de la condición despierto ("Wake"). Transformar dicha matriz de adyacencia pesada a una matriz de adyancia binaria $A_{i,j}$
que represente una una densidad de enlaces igual a 0.08. ¿Cuál es el valor de umbral de correlación entre pares de regiones que tuvo que utilizar?}

Como se observa en la figura 1, se grafica en la imagen de la izquierda la matriz de
correlaciones entre regiones. A su vez, en la imagen de la derecha, se grafica la misma
matriz representando una densidad de enlaces igual a 0.08. Para ello, se utilizó un umbral
de 0.75.

\begin{figure}[H]
  \centering  
  \includegraphics[width=1\textwidth]{1_Sujeto2W.png}
  \caption{Matriz de adyacencia pesada y binarizada del sujeto 2 despierto}
\end{figure}


\textit{5.2 Utilizando $A_{i,j}$ , obtener el grafo resultante G ¿Es G un grafo conectado? ¿Se puede calcular la distancia media entre pares de nodos
d del grafo G? ¿Si no se puede, qué medida equivalente calcularías?}

El grafo que se obtiene de $A_{i,j}$ no está conectado, por lo que no se puede calcular
la distancia media entre pares de nodos. Alternativamente, calculamos la distancia media de su
componente gigante (es decir, el componente conectadom más grande dentro del grafo G),
el cual es 3.85.

\textit{5.3 Calcular d para cada componente conectado de G. Calcular la eficiencia global ef f del
grafo G.}

La distancia d de los dos componentes conectados más grandes de G dan 3.85 y 1.28. Después,
el resto de componentes conectados tienen un valor de d igual a 0. Por último, la eficiencia
global del grafo G es 0.2446.

\textit{5.4 Obtener la lista de enlaces del grafo G.}

Ya que la lista de enlaces del grafo es extensa. Se una lista al azar de 8 enlaces del grafo G:

\begin{table}[h]
  \centering
  \caption{Primoeros 8 enlaces del grafo G.}
  \label{tab:enlaces_G}
  \begin{tabular}{rr}
    \toprule
    Nodo 1 & Nodo 2 \\
    \midrule
    28 & 96 \\
    109 & 110 \\
    73 & 75 \\
    72 & 99 \\
    16 & 57 \\
    44 & 110 \\
    8 & 22 \\
    19 & 81 \\
    \bottomrule
    \end{tabular}
\end{table}

\textit{5.5 Calcular el grado promedio < k >, el nodo con grado máximo kmax, el coeficiente de 
clustering promedio}

A continuación se muestran los valores de las medidas del grafo G en el cuadro 3:

\begin{table}[h]
  \centering
  \caption{Medidas del grafo G}
  \begin{tabular}{lr}
  \toprule
  Medida & Valor \\
  \midrule
  Grado promedio (K) & 9.206897 \\
  Nodo con grado máximo (kmax) & 30.000000 \\
  Coeficiente de clustering promedio (C) & 0.527085 \\
  Eficiencia & 0.244630 \\
  \bottomrule
  \end{tabular}
  \end{table}

\textit{5.6 Visualizar el grafo, ubicando los nodos en sus coordenadas cerebrales y coloreando cada
nodo de acuerdo a su coeficiente de clustering Ci}

\begin{figure}[H]
  \centering  
  \includegraphics[width=.5\textwidth]{2_Sujeto2W.png}
  \caption{Grafo G}
\end{figure}

\textit{5.7 Graficar la distribución de grado del grafo, elijiendo un número de bins apropriado}

\begin{figure}[H]
  \centering  
  \includegraphics[width=.5\textwidth]{3_Sujeto2W.png}
  \caption{Distribución de grado del grafo}
\end{figure}

\textit{5.8 Vamos a comparar el grafo G con prototipos de redes poissonianas (random), small-World
 y scale-free, usando los algoritmos de Erdos-Renyi, Watts-Strogatz y Barabasi-Albert, respectivamente.
 Para ello, elegir (y reportar) los parámetros utilizados para cada algoritmo, 
buscando siempre que los grafos simulados de dichos prototipos sean comparables al grafo de datos G 
(en términos de número de nodos y números de enlaces). Visualizar un ejemplo de grafo para cada uno de 
estos prototipos de redes. Discutir diferencias.}

A continuación, se enlistan los distintos parámetros utilizados para generar cada tipo
de grafo:

\begin{table}[h]
  \caption{Parámetros de las redes aleatorias}
  \begin{tabular}{lrlrll}
  \toprule
  Red & Nodos & Parámetro 2 &  & Parámetro 3 &  \\
  \midrule
  Poissoniana & 116 & Prob. creación de enlaces & 0.08 & - & - \\
  Small-world & 116 & Prob. re-conexión & 0.20 & - & - \\
  Libre de escala & 116 & N de enlaces nuevos por nodo nuevo & 4 & Prob. conexión inicial & 0.081 \\
  \bottomrule
  \end{tabular}
  \end{table}

Adicionalmente, se grafican los mismos:

\begin{figure}[H]
  \centering  
  \includegraphics[width=1\textwidth]{4_Sujeto2W.png}
  \caption{Visualización de un grafo de cada prototipo}
\end{figure}

%TODO: discutir diferencias

\textit{5.9 Generar 1000 instancias de grafos para cada uno de dichos prototipos (poissonianas,
small-World y scale-free). Para el conjunto de 1000 instancias de cada prototipo, calcular el histograma de coeficientes de < k >, kmax, C, y ef f . Comparar con los valores
de coeficientes que obtuvimos para el grafo de datos G.}

\begin{figure}[H]
  \centering  
  \includegraphics[width=1\textwidth]{5_Sujeto2W.png}
  \caption{Comparación de métricas de grafo G con el resto}
\end{figure}

%TODO: comparar con los valores de coeficientes que obtuvimos para el grafo de datos G


\end{document}